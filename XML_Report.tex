\documentclass{article} % Defines the type of document
\usepackage{geometry} % For margin adjustments
\usepackage{graphicx} % For including graphics
\usepackage{xcolor}   % For color customization
\usepackage{lipsum}   % For filler text (can be removed in actual use)

% Set custom margins for the title page
\geometry{left=2cm, right=2cm, top=3cm, bottom=3cm}

\begin{document}

% Begin title page
\begin{titlepage}
    \centering
    \vspace*{1cm}

    % Title
    {\Huge \bfseries XML Report \par}
    \vspace{1.5cm}

    % Subtitle or additional information
    {\Large A Comprehensive Exploration of XML, XPath, and XQuery \par}
    \vspace{2cm}

    % Author
    {\large Liam Power \par}
    \vspace{0.5cm}

    % Date
    {\large \today \par}

    % Optional logo or image
    \vfill
    \includegraphics[width=0.8\textwidth]{Trinity_Main_Logo.jpg} % Replace with your logo file

    % Footer note (optional)
    \vfill
    {\large This document is submitted as part of the coursework for CSU22041 Information Management I}

\end{titlepage}

% Follow-up Page with Names
\begin{newpage}
\thispagestyle{empty} % Remove page number

\vspace*{\fill} % Center vertically

\begin{center}
    {\LARGE Team Members}
    
    \vspace{1cm} % Space between title and names

    % List of names
    {\Large 
        \textbf{Finn Clancy}         23375792\\[0.5cm]
        \textbf{Robin Schulz}        22337984\\[0.5cm]
        \textbf{Liam Power}          23374796\\[0.5cm]
        \textbf{Rachel Ranjith}      23363463\\[0.5cm]
        \textbf{Nikolaos Mavrias}    24372713\\[0.5cm]
        \textbf{Roisin Smith}        23373110\\[0.5cm]
        \textbf{Noah Scolard}        23374057\\[0.5cm]
    }
\end{center}

\vspace*{\fill} % Center vertically
\end{newpage}

\newpage
\section{Approach} 
\subsection{Adjustments for Assignment Requirements}

Some classes needed adjusting to meet the assignment requirements of having six elements 
in each document. We decided to add others for depth and understanding of the whole 
design. The events document was missing two elements, so we added \texttt{manager} and 
\texttt{short description}. Similarly, the artwork document was missing two elements, so 
we added \texttt{category} and \texttt{description}. This allowed us to create detailed 
XML documents without losing the general idea of our UML design.

\subsection{XQuery Ideas}

When coming up with ideas for interesting XQueries, we thought it would be beneficial to 
have a query to find out how many pieces of art the gallery currently has that fit in a 
specific category, such as "Post-Impressionism." This query will make it easier for 
curators to gauge if there are enough pieces of art available for their desired exhibition. 
This update forced us to adjust our artwork class to include a category variable to match 
the category elements in the XML DTD.

Another idea for an interesting XQuery was to have different ways of structuring important 
events. An event ID and a manager were mentioned. The ID existed in the use case diagram, 
but the manager did not. By adding this, we not only created the XML documents and their 
XQueries according to the assignment guidelines but also made our use cases more 
comprehensive. For instance, in this example, the relationship between external partners, 
events, and the administration of the Art Gallery was strengthened by adding a gallery 
manager overseeing the events.

\subsection{Artist Class Adjustments}

We added the following variables to the \texttt{Artist} class to ensure it matched the 
elements in the Artist XML DTD: \texttt{deathYear}, \texttt{specialty}, and 
\texttt{placeOfStudy}.

\subsection{Curator Use Case}

We also added a use case for the Curator called “Get Artwork For Each Artist,” which 
will make the Curator's workflow more efficient. This allows them to gather information 
for potential exhibitions quickly or respond to inquiries from auctioneers, museums, and 
other galleries.

\subsection{Group Processes}
We held group meetings during the tutorial times to coordinate our efforts and ensure 
everyone was on the same page regarding project tasks. Meetings were scheduled during 
and after lecture times to maximize participation and accommodate different schedules. 
We brainstormed possible XML documents and XQueries to explore various approaches and 
solutions for our project.

The class diagram dimension was adjusted for the XML tasks to better align with the 
project requirements and improve design accuracy. Tasks were allocated among team 
members, and progress was discussed regularly to track our development and address any 
issues promptly. There was in-person collaboration on XML documents, which facilitated 
direct communication and quick problem-solving. Virtual collaboration occurred through 
a GitHub repository, allowing us to share updates efficiently and work asynchronously.

\section{UML Design to XML Implementation}
Some classes needed adjusting to meet the assignment requirements of having six 
elements in each document. We decided to add additional elements for depth and better 
understanding of the overall design. The events document was missing two elements, so 
we added "manager" and "short description." Similarly, the artwork document was 
missing two elements, so we added "category" and "description." These additions 
allowed us to create detailed XML documents without losing the general idea of our UML 
design.

When brainstorming ideas for interesting XQueries, we thought it would be beneficial 
to have a query that determines how many pieces of art the gallery currently has in 
a specific category, such as "Post-Impressionism." This query will make it easier for 
curators to gauge if there are enough pieces of art available for their desired 
exhibition. This update required us to adjust our artwork class to include a category 
variable to match the category elements in the XML DTD.

Another idea for an interesting XQuery was to explore different ways of structuring 
important events. An event ID and a manager were mentioned. The ID existed in the use 
case diagram, but the manager did not. By adding this, we were able to create the XML 
documents and their XQueries according to the assignment guidelines while also making 
our use cases more comprehensive. For instance, in this example, the relationship 
between external partners, events, and the administration of the Art Gallery was 
strengthened by adding a gallery manager overseeing the events.

We added the following variables to the "Artist" class to ensure it matched the 
elements in the Artist XML DTD: deathYear, specialty, and placeOfStudy.

We also added a use case for the Curator called “Get Artwork For Each Artist,” which 
will make the Curator's workflow more efficient. This allows them to gather 
information for potential exhibitions quickly or respond to inquiries from auctioneers, 
museums, and other galleries.

\section{Strengths and Weaknesses Analysis}

\subsection{XML}

\subsubsection{Strengths}
\begin{itemize}
    \item We used a standard to create our documents, ensuring consistency across the project.
    \item The individual XML documents were evenly distributed among team members, promoting balanced workloads.
    \item The use of modular XML designs allows for easier future updates or modifications.
\end{itemize}

\subsubsection{Weaknesses}
\begin{itemize}
    \item Time constraints during the design phase limited our ability to fully evaluate which classes would work best as XML documents.
    \item Although we gained significant knowledge, not all team members could immediately understand the concepts due to the lack of centralized documentation.
    \item The large number of Classes in the Class Diagram made it challenging to decide which ones to convert into XML files.
\end{itemize}

\subsection{XQuery}

\subsubsection{Strengths}
\begin{itemize}
    \item Consistent use of variables, such as dates and prices, across different queries ensured clarity and standardization.
    \item XQueries were assigned based on prior involvement with specific XML documents, keeping team members focused on familiar Classes.
    \item The XQueries spanned multiple XML documents, providing a rich and interconnected dataset for queries.
\end{itemize}

\subsubsection{Weaknesses}
\begin{itemize}
    \item The large number of XML files created made it difficult to decide which ones to include in the XQueries.
    \item While designing XQueries, parts of the Class Diagram had to be reconfigured to connect certain Classes, introducing additional complexity.
    \item Traversing different Classes with XQueries was challenging because different team members used inconsistent element names for attributes like dates and prices.
\end{itemize}

\newpage
\section{XQueries}
\subsection{Query for Total Artworks Per Artist}
\begin{itemize}
    \item \textbf{Identification of the UML Use Case:} Supports the ``Curators'' use case.
    \item \textbf{Description of the Purpose:} This query allows curators to determine how many pieces of art the gallery holds for each artist. It helps in planning exhibitions for specific anniversaries and assessing whether additional pieces need to be sourced from other institutions.
\end{itemize}

\textbf{Example Output:}
\begin{verbatim}
<artistTotal>
    <name>Vincent van Gogh</name>
    <totalArtworks>3</totalArtworks>
</artistTotal>
\end{verbatim}

\subsection{Query for Events Managed by a Specific Manager}
\begin{itemize}
    \item \textbf{Identification of the UML Use Case:} Supports the ``Creates Event'' use case.
    \item \textbf{Description of the Purpose:} This query identifies all events managed by a specific manager based on their first name. It provides a comprehensive list of events for planning and oversight purposes.
\end{itemize}

\textbf{Example Output:}  
Input: \texttt{Robin}
\begin{verbatim}
<is_manager_for>
    <eventName>Historical Easter Getaway</eventName>
</is_manager_for>
<is_manager_for>
    <eventName>Boxing Day Charity</eventName>
</is_manager_for>
<is_manager_for>
    <eventName>Van Gogh Week</eventName>
</is_manager_for>
<is_manager_for>
    <eventName>Photography Festival</eventName>
</is_manager_for>
<is_manager_for>
    <eventName>Sponsor Gratitude Evening</eventName>
</is_manager_for>
\end{verbatim}

Input: \texttt{Liam}
\begin{verbatim}
<is_manager_for>
    <eventName>Children's Art Workshop</eventName>
</is_manager_for>
\end{verbatim}

\subsection{Query for Event Duration}
\begin{itemize}
    \item \textbf{Identification of the UML Use Case:} Supports the ``Creates Event'' use case.
    \item \textbf{Description of the Purpose:} This query calculates and outputs the duration of a specified event in days, including the specific dates it spans.
\end{itemize}

\textbf{Example Output:}  
Input: \texttt{Van Gogh Week}  
\begin{verbatim}
Van Gogh Week lasts 7 days from the 24th to the 30th.
\end{verbatim}

Input: \texttt{Sponsor Gratitude Evening}  
\begin{verbatim}
Sponsor Gratitude Evening lasts 1 day and is on the 12th.
\end{verbatim}

\subsection{Query for Membership Revenue}
\begin{itemize}
    \item \textbf{Identification of the UML Use Case:} Supports the ``Purchase Membership'' use case.
    \item \textbf{Description of the Purpose:} This query calculates the total revenue generated from each type of membership. It helps assess revenue performance and informs membership-related decision-making.
\end{itemize}

\textbf{Example Output:}
\begin{verbatim}
<TotalMembershipRevenue type="Patron_Director">5000.0</TotalMembershipRevenue>
<TotalMembershipRevenue type="Patron_Dargan">10000.0</TotalMembershipRevenue>
<TotalMembershipRevenue type="Patron_Curator">4500.0</TotalMembershipRevenue>
\end{verbatim}

\subsection{Query for Artworks by Category}
\begin{itemize}
    \item \textbf{Identification of the UML Use Case:} Supports the ``Creates Event'' use case.
    \item \textbf{Description of the Purpose:} This query allows curators to see the number of artworks available in each category, helping them organize exhibitions based on specific themes or styles.
\end{itemize}

\textbf{Example Output:}
\begin{verbatim}
<category>
    <name>Post-Impressionism</name>
    <artworkCount>3</artworkCount>
</category>
<category>
    <name>Surrealism</name>
    <artworkCount>4</artworkCount>
</category>
<category>
    <name>Expressionism</name>
    <artworkCount>1</artworkCount>
</category>
<category>
    <name>Impressionism</name>
    <artworkCount>1</artworkCount>
</category>
\end{verbatim}

\subsection{Query for Feedback from Non-Paying Visitors}
\begin{itemize}
    \item \textbf{Identification of the UML Use Case:} Supports the ``Feedback'' use case.
    \item \textbf{Description of the Purpose:} This query retrieves feedback submitted by non-paying visitors, potentially identifying spam or malicious reviews.
\end{itemize}

\textbf{Example Output:}
\begin{verbatim}
<FeedbackFromNonPayingUser>
    <visitorID>V123</visitorID>
    <rating>1</rating>
    <comments>The reception staff is rude, paintings boring</comments>
    <dateSubmitted>2023-11-14</dateSubmitted>
    <ratingValue>1</ratingValue>
</FeedbackFromNonPayingUser>
\end{verbatim}

\subsection{Query for Donor Contributions}
\begin{itemize}
    \item \textbf{Identification of the UML Use Case:} Supports the ``Donate'' use case.
    \item \textbf{Description of the Purpose:} This query retrieves and displays donations made by a specific donor, including details such as amount, currency, and donation type.
\end{itemize}

\textbf{Example Output:}
\begin{verbatim}
<DonorDonation>
    <DonationID>D001</DonationID>
    <Amount>100.00</Amount>
    <Currency>EUR</Currency>
    <DonationType>Individual</DonationType>
</DonorDonation>
\end{verbatim}

\subsection{Query for Total Revenue on a Specific Date}
\begin{itemize}
    \item \textbf{Identification of the UML Use Case:} Supports the ``Donations,'' ``Memberships,'' and ``Tickets'' use cases.
    \item \textbf{Description of the Purpose:} This query calculates total revenue generated from donations, memberships, and tickets on a specific date.
\end{itemize}

\textbf{Example Output:}  
Input: \texttt{2023-11-15}  
\begin{verbatim}
<TotalRevenue>The total revenue for 2023-11-15 is $8,390.00!</TotalRevenue>
\end{verbatim}

\newpage
\section{XML Documents}

\subsection{Artist}

\begin{itemize}
    \item Finn
\end{itemize}

\begin{verbatim}
<?xml version="1.0" ?>

<!-- 
This XML document stores information about artists, including their biographical details, 
artistic specialty, and place of study. It can be used in applications or queries to 
retrieve artist-specific data.
-->

<!DOCTYPE artists [
  
  <!-- 
  The "artists" element contains one or more "artist" elements, 
  hence the cardinality is "+" (at least one artist is required). 
  -->
  <!ELEMENT artists (artist+)>
  
  <!-- 
  Each "artist" includes details about their name, nationality, birth and death years, 
  artistic specialty, and place of study. These elements are all required, 
  so no optional or zero cardinality is specified.
  -->
  <!ELEMENT artist (name, nationality, birthYear, deathYear, specialty, placeOfStudy)>
  
  <!ELEMENT name (#PCDATA)> <!-- The artist's name -->
  <!ELEMENT nationality (#PCDATA)> <!-- The artist's nationality -->
  <!ELEMENT birthYear (#PCDATA)> <!-- The year the artist was born -->
  <!ELEMENT deathYear (#PCDATA)> <!-- The year the artist passed away -->
  <!ELEMENT specialty (#PCDATA)> <!-- The artist's area of specialization -->
  <!ELEMENT placeOfStudy (#PCDATA)> <!-- The institution where the artist studied -->
]>

<artists>
  <!-- An example of an artist's information -->
  <artist>
    <name>Vincent van Gogh</name>
    <nationality>Dutch</nationality>
    <birthYear>1853</birthYear>
    <deathYear>1890</deathYear>
    <specialty>Post-Impressionist</specialty>
    <placeOfStudy>Académie Royale des Beaux-Arts</placeOfStudy>
  </artist>
  <!-- Additional artist records follow the same structure -->
</artists>

\end{verbatim}

\subsection{Events}

\begin{itemize}
    \item Robin
\end{itemize}

\begin{verbatim}
<?xml version="1.0" encoding="UTF-8"?>

<!--This is the DTD and the XML document for the events class and is used by two event 
related XQueries-->

<!DOCTYPE events [

<!--The following is a description of the main elements, the cardinality and the general 
outline of the DTD seen below:

Event is set to "*: zero-or-more" as no events are technically needed. There will not 
always be an event planned to document. All of the event elements are set to one as we 
at least need all the information in the document for each single event once. By using
an extra outside bracket for dates, we are not limited in the amount of dates for each 
event, hence why it is "+: one-or-more". There is at least one location needed. Some 
events might take multiple places/rooms but one is the minimum.

The 6 #PCData elements are all elements in the document helping to describe the events.

eventID ATTLIST is #REQUIRED for every new event for easier allocation and logistic 
purposes-->

<!ELEMENT events (event*)>		
<!ELEMENT event (eventName, dates, locations, host, manager, eventDescription)>		
<!ELEMENT dates (date+)>		
<!ELEMENT locations (location+)>	

<!ELEMENT eventName (#PCDATA)>		
<!ELEMENT host (#PCDATA)>
<!ELEMENT date (#PCDATA)>
<!ELEMENT location (#PCDATA)>
<!ELEMENT manager (#PCDATA)>
<!ELEMENT eventDescription (#PCDATA)>

<!ATTLIST event eventID CDATA #REQUIRED>  	

]>

<!--The following XML document is used to give functionality to the events part of our 
UML diagram.

The diagrams showed an external partner working with the admins to create and manage 
events, so in this document the functionality of a system is created,that allows the 
management to oversee certain events with all the necessary details. This could allow 
them to properly prepare, to have a general overview, to structure the year, to 
calculate funds and to make the necessary changes needed.
    
We have the events document, with the individual events all sorted via eventID. Within 
these events, there are multiple other elements that had mostly been included in the 
class diagram already. Some additional events were created. Every event has its eventName, 
its dates, the location it will be occupying/using, the host/external partner, the manager 
responsible for the event and a short event description. All of this information allows 
everyone that has access, to get a quick overview of the most important facts and to 
understand the dimensions of each event.

    There are 8 events with different data to be found below. -->

<events>
    <event eventID="230401001">
        <eventName>Historical Easter Getaway</eventName>
        <dates>
            <date>2023-04-01</date>
        </dates>
        <locations>
            <location>Front Court</location>
        </locations>
        <host>Trinity Food Society</host>
        <manager>Robin Schulz</manager>
        <eventDescription>
The Historical Easter Getaway is an event focused on exploring and celebrating the 
cultural and historical traditions of Easter, featuring themed activities and 
educational presentations.
        </eventDescription>
    </event>
    <!-- Additional events records follow the same structure -->
</events>
    
\end{verbatim}

\subsection{Membership}

\begin{itemize}
    \item Liam
\end{itemize}

\begin{verbatim}
<?xml version="1.0" encoding="UTF-8"?>

<!-- 
    This XML document represents a list of memberships for an organisation or institution. 
    Each membership contains details e.g. purchase price, type, purchase data, expiration, 
    date, visitor ID and membership type
-->

<!DOCTYPE Members [

<!ELEMENT Members (Membership+)> <!-- The '+' cardinality is used to indicate that there 
must be at least one 'Membership' element in the document-->
<!ELEMENT Membership (price, expirationDate, visitorID, type)> <!-- Each 'Membership' 
element contains child elements -->
<!ELEMENT purchasePrice (#PCDATA)>
<!ELEMENT purchaseType (#PCDATA)>
<!ELEMENT purchaseDate (#PCDATA)>
<!ELEMENT expirationDate (#PCDATA)>
<!ELEMENT visitorID (#PCDATA) >
<!ELEMENT type (#PCDATA)>

<!ATTLIST Membership membershipID CDATA #REQUIRED> <!-- The 'membershipID' attribute is 
required for each 'Membership' element to uniquely identify it-->

]>


<Members>
    <!-- Membership record for Visitor ID 1234 with a Patron membership-->
    <Membership membershipID = "0001">
        <purchasePrice>500</purchasePrice>
        <purchaseType>Final</purchaseType>
        <purchaseDate>2023-11-14</purchaseDate>
        <expirationDate>2024-11-14</expirationDate>
        <visitorID>1234</visitorID>
        <type>Patron</type>
    </Membership>
    <!-- Additional membership records follow the same structure -->
</Members>
    
\end{verbatim}

\subsection{Artwork}

\begin{itemize}
    \item Rachel
\end{itemize}

\begin{verbatim}
<?xml version="1.0" ?>

<!-- This XML document represents a museum's collection of artworks. 
        Each artwork entry contains detailed information about the piece,
        including its title, artist, submission date, status, category, and 
        description. -->

<!DOCTYPE museum [

    <!-- The root element 'museum' contains multiple 'artwork' elements. 
            Cardinality '*' is used to allow zero or more artworks. -->
    <!ELEMENT museum (artwork*)>

    <!-- The 'artwork' element represents an individual art piece.
            Each artwork must have exactly one of each of the following child 
            elements: title, artistName, submissionDate, status, category, and 
            description. -->
    <!ELEMENT artwork (title, artistName, submissionDate, status, category, 
    description)>

    <!-- Simple text content (#PCDATA) is expected for each child element. -->
    <!ELEMENT title (#PCDATA)>
    <!ELEMENT artistName (#PCDATA)>
    <!ELEMENT submissionDate (#PCDATA)>
    <!ELEMENT status (#PCDATA)>
    <!ELEMENT category (#PCDATA)>
    <!ELEMENT description (#PCDATA)>

    <!-- The attribute 'artworkID' is optional (IMPLIED) and is used as a 
    unique identifier for each artwork. -->
    <!ATTLIST artwork artworkID CDATA #IMPLIED>
]>

<museum>
    <artwork artworkID="2001">
    <title>Starry Night</title>
    <artistName>Vincent van Gogh</artistName>
    <submissionDate>12/06/2024</submissionDate>
    <status>Displayed</status>
    <category>Post-Impressionism</category>
    <description>
        A depiction of a swirling night sky filled with stars and expressive 
        brushwork that conveys intense emotion.
    </description>
    <!-- Additional artwork entries follow the same structure -->
</museum>
\end{verbatim}

\subsection{Tickets}
\begin{itemize}
    \item Noah
\end{itemize}

\begin{verbatim}
<?xml version="1.0" ?>

<!-- 
This is the XML and DTD document for the Tickets Use Case. It stores IDs for 
both tickets and visitors, the type of tickets along with their price, purchase 
date, status and expiration date. -->

<!DOCTYPE Tickets [
    
    <!-- “Tickets” element has an undefined number of elements, but at least more  
    than one, hence the “+”. -->
    <!ELEMENT Tickets (Ticket+)>

    <!-- All tickets include information about their type of ticket, along with 
    prices and validity dates. --> 
    <!ELEMENT Ticket (visitorID, ticketType, price, purchaseDate, ticketStatus, 
    expirationDate)>

    <!-- every ticket has a mandatory attribute that must be included, being 
    the ID of the ticket -->
    <!ATTLIST Ticket ticketID ID #REQUIRED>

    <!ELEMENT visitorID (#PCDATA)> <!-- ID of the visitor -->
    <!ELEMENT ticketType (#PCDATA)> <!-- type of ticket -->
    <!ELEMENT price (#PCDATA)> <!-- price of ticket -->
    <!ELEMENT purchaseDate (#PCDATA)> <!-- date of purchase -->
    <!ELEMENT ticketStatus (#PCDATA)> <!-- validity of ticket -->
    <!ELEMENT expirationDate (#PCDATA)> <!-- expiration date of ticket -->
    ]>

<Tickets>
    <!-- one example of a ticket object -->
    <Ticket ticketID=”001”>
    <visitorID>v150</visitorID>
    <ticketType>Adult</ticketType>
    <price>75.00</price>
    <purchaseDate>2023-11-14</purchaseDate>
    <ticketStatus>Expired</ticketStatus>
    <expirationDate>2023-11-16</expirationDate>
    </Ticket>
    <!-- following ticket objects have the exact same structure -->
</Tickets>
    
\end{verbatim}

\subsection{Donations}
\begin{itemize}
    \item Roisin
\end{itemize}

\begin{verbatim}
<?xml version="1.0" encoding="UTF-8"?>

<!--  This XML document stores donation information for various events. Each 
donation record includes details about the donor, the amount donated, payment 
date, type of donation, and the purpose of the donation.
→

<!DOCTYPE Donations [
    <!-- The 'Donations' element contains one or more 'Donation' elements -->
    <!ELEMENT Donations (Donation+)>

    <!-- The 'Donation' element must contain six child elements: donationID, 
    donorID, amount, paymentDate, donationType, donationPurpose -->
    <!ELEMENT Donation (donationID, donorID, amount, paymentDate, donationType, 
    donationPurpose)>

    <!-- 'donationID' is a simple text-based element that uniquely identifies the 
    donation -->
    <!ELEMENT donationID (#PCDATA)>

    <!-- 'donorID' is a simple text-based element that identifies the donor -->
    <!ELEMENT donorID (#PCDATA)>

    <!-- 'amount' is a text-based element that contains the donation amount, with 
    a required attribute 'currency' specifying the currency -->
    <!ELEMENT amount (#PCDATA)>
    <!ATTLIST amount currency CDATA #REQUIRED>

    <!-- 'paymentDate' is a text-based element that specifies the date of the 
    donation -->
    <!ELEMENT paymentDate (#PCDATA)>

    <!-- 'donationType' specifies the type of the donor (e.g., Individual, 
    Corporate, Organization) -->
    <!ELEMENT donationType (#PCDATA)>

    <!-- 'donationPurpose' specifies the purpose of the donation (e.g., Exhibition 
    Support, General Fund) -->
    <!ELEMENT donationPurpose (#PCDATA)>
]

<Donations>
    <!-- Each Donation represents a single donation made to the gallery -->
    <Donation>
        <donationID>D001</donationID>
        <!-- Unique identifier for the donation -->
        <donorID>USER001</donorID>
        <!-- Unique identifier for the donor -->
        <amount currency="EUR">100.00</amount>
        <!-- The amount donated, with the associated currency (EUR) -->
        <paymentDate>2023-11-14</paymentDate>
        <!-- The date on which the donation was made -->
        <donationType>Individual</donationType>
        <!-- Type of donor: Individual, Corporate, Organization -->
        <donationPurpose>Exhibition Support</donationPurpose>
        <!-- The purpose for which the donation was made -->
    </Donation>
    <!-- Additional Donation records follow the same structure -->
</Donations>
    
\end{verbatim}

\subsection{Payment Systems}
\begin{itemize}
    \item Nikolaos
\end{itemize}

\begin{verbatim}
<?xml version="1.0" encoding="UTF-8"?>
<!DOCTYPE PaymentSystems [
    <!ELEMENT PaymentSystems (PaymentSystem+)>
    <!ELEMENT PaymentSystem (amount, paymentDate, paymentMethod, userID, status)>
    <!ATTLIST PaymentSystem paymentID ID #REQUIRED>
    <!ELEMENT amount (#PCDATA)>
    <!ELEMENT paymentDate (#PCDATA)>
    <!ELEMENT paymentMethod (#PCDATA)>
    <!ELEMENT userID (#PCDATA)>
    <!ELEMENT status (#PCDATA)>
]>
<PaymentSystems>
    <PaymentSystem paymentID="p002">
        <amount>75.00</amount>
        <paymentDate>2023-11-15</paymentDate>
        <paymentMethod>PayPal</paymentMethod>
        <userID>v124</userID>
        <status>completed</status>
    </PaymentSystem>
    <PaymentSystem paymentID="p003">
        <amount>30.00</amount>
        <paymentDate>2023-11-16</paymentDate>
        <paymentMethod>Debit Card</paymentMethod>
        <userID>v125</userID>
        <status>completed</status>
    </PaymentSystem>
    <PaymentSystem paymentID="p004">
        <amount>35.00</amount>
        <paymentDate>2022-11-16</paymentDate>
        <paymentMethod>Debit Card</paymentMethod>
        <userID>v126</userID>
        <status>completed</status>
    </PaymentSystem>
</PaymentSystems>
\end{verbatim}

\subsection{Visitor Users}
\begin{itemize}
    \item Nikolaos
\end{itemize}

\begin{verbatim}
<?xml version="1.0" encoding="UTF-8"?>
<!DOCTYPE VisitorUsers [
    <!ELEMENT VisitorUsers (VisitorUser+)>
    <!ELEMENT VisitorUser (name, email, membershipStatus, purchases, feedbackID)>
    <!ATTLIST VisitorUser visitorID ID #REQUIRED>
    <!ELEMENT name (#PCDATA)>
    <!ELEMENT email (#PCDATA)>
    <!ELEMENT membershipStatus (#PCDATA)>
    <!ELEMENT purchases (ticketID+)>
    <!ELEMENT ticketID (#PCDATA)>
    <!ELEMENT feedbackID (#PCDATA)>
]>
<VisitorUsers>
    <VisitorUser visitorID="v123">
        <name>Alex Turner</name>
        <email>alex505@boardwalk.com</email>
        <membershipStatus>false</membershipStatus>
        <purchases>
            <ticketID></ticketID>
        </purchases>
        <feedbackID>f100</feedbackID>
    </VisitorUser>
    <VisitorUser visitorID="v124">
        <name>Jane Smith</name>
        <email>jane.smith@example.com</email>
        <membershipStatus>false</membershipStatus>
        <purchases>
            <ticketID>t003</ticketID>
        </purchases>
        <feedbackID>f101</feedbackID>
    </VisitorUser>
    <VisitorUser visitorID="v125">
        <name>Alex Johnson</name>
        <email>alex.johnson@example.com</email>
        <membershipStatus>true</membershipStatus>
        <purchases>
            <ticketID>t004</ticketID>
        </purchases>
        <feedbackID>f102</feedbackID>
    </VisitorUser>
    <VisitorUser visitorID="v126">
        <name>Lorem Ipsum</name>
        <email>lorem.ipsum@example.com</email>
        <membershipStatus>true</membershipStatus>
        <purchases>
            <ticketID>t005</ticketID>
        </purchases>
        <feedbackID>f105</feedbackID>
    </VisitorUser>
</VisitorUsers>
\end{verbatim}

\subsection{Feedbacks}
\begin{itemize}
    \item Nikolaos
\end{itemize}

\begin{verbatim}
<?xml version="1.0" encoding="UTF-8"?>
<!DOCTYPE Feedbacks [
    <!ELEMENT Feedbacks (Feedback+)>
    <!ELEMENT Feedback (visitorID, rating, comments, dateSubmitted, ratingValue)>
    <!ATTLIST Feedback feedbackID ID #REQUIRED>
    <!ELEMENT visitorID (#PCDATA)>
    <!ELEMENT rating (#PCDATA)>
    <!ELEMENT comments (#PCDATA)>
    <!ELEMENT dateSubmitted (#PCDATA)>
    <!ELEMENT ratingValue (#PCDATA)>
]>
<Feedbacks>
    <Feedback feedbackID="f100">
        <visitorID>v123</visitorID>
        <rating>1</rating>
        <comments>The reception staff is rude, paintings boring</comments>
        <dateSubmitted>2023-11-14</dateSubmitted>
        <ratingValue>1</ratingValue>
    </Feedback>
    <Feedback feedbackID="f101">
        <visitorID>v124</visitorID>
        <rating>5</rating>
        <comments>Loved it! Highly recommend.</comments>
        <dateSubmitted>2023-11-15</dateSubmitted>
        <ratingValue>5</ratingValue>
    </Feedback>
    <Feedback feedbackID="f102">
        <visitorID>v125</visitorID>
        <rating>3</rating>
        <comments>It was good, but could be better.</comments>
        <dateSubmitted>2023-11-16</dateSubmitted>
        <ratingValue>3</ratingValue>
    </Feedback>
    <Feedback feedbackID="f103">
        <visitorID>v126</visitorID>
        <rating>5+</rating>
        <comments>Good ol craic</comments>
        <dateSubmitted>2023-11-16</dateSubmitted>
        <ratingValue>5+</ratingValue>
    </Feedback>
</Feedbacks>
\end{verbatim}

\newpage
\section{Credits}

\subsection{Finn}
\begin{itemize}
    \item Created 1 XML document: Artist
    \item Added six elements
    \item Created 1 XQuery:
    \begin{description}
        \item [artworkPerArtist:] Outputs the amount of artwork the gallery has per artist.
    \end{description}
    \item Established a GitHub to enable a collaborative workflow to store XML documents.
\end{itemize}

\subsection{Robin}
\begin{itemize}
    \item Created 1 XML document: Events
    \item Added six elements 
    \item Created 2 XQueries:
    \begin{description}
        \item[eventsLengthInDays:] Takes in an event title and prints out the number of days and which days of the month the event occurs.
        \item[eventsByManager:] Takes in a manager's first name and prints out all the events this manager is responsible for.
    \end{description}
    \item Created a poll for task allocation
    \item Created the XML report template
    \item Contributed to:
    \begin{description}
        \item UML Design to XML Implementation
        \item XQueries
        \item XML Documents
    \end{description}
\end{itemize}

\subsection{Liam}
\begin{itemize}
    \item Created 1 XML document: Membership
    \item Added six elements
    \item Created 1 XQuery: 
    \begin{description}
        \item [getMemberRevenue:] Calculates and displays total revenue generated from types of memberships.
    \end{description}
    \item Compiled and formatted the final report
\end{itemize}

\subsection{Rachel}
\begin{itemize}
    \item Created 1 XML document: Artwork
    \item Added two elements
    \item Created 1 XQuery: 
    \begin{description}
        \item[getArtworkForCategory:] Implements user-defined function
    \end{description}
\end{itemize}

\subsection{Nick}
\begin{itemize}
    \item Created 3 XML documents: Feedback, PaymentSystem and UserVisitor
    \item Added multiple elements 
    \item Created 2 XQueries: 
    \begin{description}
        \item[getFeedbackFromNonPayingUsers:] Retrieves feedback from non-paying visitors.
        \item[getFeedbackFromPayingUsers:] Retrieves feedback from paying visitors.
    \end{description}
\end{itemize}

\subsection{Roisin}
\begin{itemize}
    \item Created 1 XML document: Donation
    \item Added six elements 
    \item Created 1 XQuery: 
    \begin{description}
        \item[getSpecificDonor:] Retrieves and displays donations by a specific donor.
    \end{description}
    \item Identified strengths and weaknesses of the XML and XQuery design.
\end{itemize}

\subsection{Noah}
\begin{itemize}
    \item Created 1 XML Document: Tickets
    \item Added six elements 
    \item Created 1 XQuery: 
    \begin{description}
        \item[getRevenueFromDate:] Implements user-defined function that retrieves all revenue across Tickets, Membership, and Donations on a certain date.
    \end{description}
    \item Identified strengths and weaknesses of the XML and XQuery design.
\end{itemize}

\end{document}            % Ends the document content