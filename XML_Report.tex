\documentclass{article} % Defines the type of document
\usepackage{geometry} % For margin adjustments
\usepackage{graphicx} % For including graphics
\usepackage{xcolor}   % For color customization
\usepackage{lipsum}   % For filler text (can be removed in actual use)

% Set custom margins for the title page
\geometry{left=2cm, right=2cm, top=3cm, bottom=3cm}

\begin{document}

% Begin title page
\begin{titlepage}
    \centering
    \vspace*{1cm}

    % Title
    {\Huge \bfseries XML Report \par}
    \vspace{1.5cm}

    % Subtitle or additional information
    {\Large A Comprehensive Exploration of XML, XPath, and XQuery \par}
    \vspace{2cm}

    % Author
    {\large Liam Power \par}
    \vspace{0.5cm}

    % Date
    {\large \today \par}

    % Optional logo or image
    \vfill
    \includegraphics[width=0.8\textwidth]{Trinity_Main_Logo.jpg} % Replace with your logo file

    % Footer note (optional)
    \vfill
    {\large This document is submitted as part of the coursework for CSU22041 Information Management I}

\end{titlepage}

% Follow-up Page with Names
\begin{newpage}
\thispagestyle{empty} % Remove page number

\vspace*{\fill} % Center vertically

\begin{center}
    {\LARGE Team Members}
    
    \vspace{1cm} % Space between title and names

    % List of names
    {\Large 
        Finn Clancy \\[0.5cm]
        Robin Schulz \\[0.5cm]
        Liam Power \\[0.5cm]
        Rachel  \\[0.5cm]
        Nick  \\[0.5cm]
        Roisin  \\[0.5cm]
        Noah Scolard \\[0.5cm]
    }
\end{center}

\vspace*{\fill} % Center vertically
\end{newpage}

\newpage
\section{Approach} 
\subsection{Adjustments for Assignment Requirements}

Some classes needed adjusting to meet the assignment requirements of having six elements 
in each document. We decided to add others for depth and understanding of the whole 
design. The events document was missing two elements, so we added \texttt{manager} and 
\texttt{short description}. Similarly, the artwork document was missing two elements, so 
we added \texttt{category} and \texttt{description}. This allowed us to create detailed 
XML documents without losing the general idea of our UML design.

\subsection{XQuery Ideas}

When coming up with ideas for interesting XQueries, we thought it would be beneficial to 
have a query to find out how many pieces of art the gallery currently has that fit in a 
specific category, such as "Post-Impressionism." This query will make it easier for 
curators to gauge if there are enough pieces of art available for their desired exhibition. 
This update forced us to adjust our artwork class to include a category variable to match 
the category elements in the XML DTD.

Another idea for an interesting XQuery was to have different ways of structuring important 
events. An event ID and a manager were mentioned. The ID existed in the use case diagram, 
but the manager did not. By adding this, we not only created the XML documents and their 
XQueries according to the assignment guidelines but also made our use cases more 
comprehensive. For instance, in this example, the relationship between external partners, 
events, and the administration of the Art Gallery was strengthened by adding a gallery 
manager overseeing the events.

\subsection{Artist Class Adjustments}

We added the following variables to the \texttt{Artist} class to ensure it matched the 
elements in the Artist XML DTD: \texttt{deathYear}, \texttt{specialty}, and 
\texttt{placeOfStudy}.

\subsection{Curator Use Case}

We also added a use case for the Curator called “Get Artwork For Each Artist,” which 
will make the Curator's workflow more efficient. This allows them to gather information 
for potential exhibitions quickly or respond to inquiries from auctioneers, museums, and 
other galleries.

\subsection{Group Processes}
We held group meetings during the tutorial times to coordinate our efforts and ensure 
everyone was on the same page regarding project tasks. Meetings were scheduled during 
and after lecture times to maximize participation and accommodate different schedules. 
We brainstormed possible XML documents and XQueries to explore various approaches and 
solutions for our project.

The class diagram dimension was adjusted for the XML tasks to better align with the 
project requirements and improve design accuracy. Tasks were allocated among team 
members, and progress was discussed regularly to track our development and address any 
issues promptly. There was in-person collaboration on XML documents, which facilitated 
direct communication and quick problem-solving. Virtual collaboration occurred through 
a GitHub repository, allowing us to share updates efficiently and work asynchronously.

\section{UML Design to XML Implementation}
Some classes needed adjusting to meet the assignment requirements of having six 
elements in each document. We decided to add additional elements for depth and better 
understanding of the overall design. The events document was missing two elements, so 
we added "manager" and "short description." Similarly, the artwork document was 
missing two elements, so we added "category" and "description." These additions 
allowed us to create detailed XML documents without losing the general idea of our UML 
design.

When brainstorming ideas for interesting XQueries, we thought it would be beneficial 
to have a query that determines how many pieces of art the gallery currently has in 
a specific category, such as "Post-Impressionism." This query will make it easier for 
curators to gauge if there are enough pieces of art available for their desired 
exhibition. This update required us to adjust our artwork class to include a category 
variable to match the category elements in the XML DTD.

Another idea for an interesting XQuery was to explore different ways of structuring 
important events. An event ID and a manager were mentioned. The ID existed in the use 
case diagram, but the manager did not. By adding this, we were able to create the XML 
documents and their XQueries according to the assignment guidelines while also making 
our use cases more comprehensive. For instance, in this example, the relationship 
between external partners, events, and the administration of the Art Gallery was 
strengthened by adding a gallery manager overseeing the events.

We added the following variables to the "Artist" class to ensure it matched the 
elements in the Artist XML DTD: deathYear, specialty, and placeOfStudy.

We also added a use case for the Curator called “Get Artwork For Each Artist,” which 
will make the Curator's workflow more efficient. This allows them to gather 
information for potential exhibitions quickly or respond to inquiries from auctioneers, 
museums, and other galleries.

\section{XQueries}

\section{XML Documents}

\subsection{Artist}

\begin{itemize}
    \item Finn
\end{itemize}

\begin{verbatim}
<?xml version="1.0" ?>

<!-- 
This XML document stores information about artists, including their biographical details, 
artistic specialty, and place of study. It can be used in applications or queries to 
retrieve artist-specific data.
-->

<!DOCTYPE artists [
  
  <!-- 
  The "artists" element contains one or more "artist" elements, 
  hence the cardinality is "+" (at least one artist is required). 
  -->
  <!ELEMENT artists (artist+)>
  
  <!-- 
  Each "artist" includes details about their name, nationality, birth and death years, 
  artistic specialty, and place of study. These elements are all required, 
  so no optional or zero cardinality is specified.
  -->
  <!ELEMENT artist (name, nationality, birthYear, deathYear, specialty, placeOfStudy)>
  
  <!ELEMENT name (#PCDATA)> <!-- The artist's name -->
  <!ELEMENT nationality (#PCDATA)> <!-- The artist's nationality -->
  <!ELEMENT birthYear (#PCDATA)> <!-- The year the artist was born -->
  <!ELEMENT deathYear (#PCDATA)> <!-- The year the artist passed away -->
  <!ELEMENT specialty (#PCDATA)> <!-- The artist's area of specialization -->
  <!ELEMENT placeOfStudy (#PCDATA)> <!-- The institution where the artist studied -->
]>

<artists>
  <!-- An example of an artist's information -->
  <artist>
    <name>Vincent van Gogh</name>
    <nationality>Dutch</nationality>
    <birthYear>1853</birthYear>
    <deathYear>1890</deathYear>
    <specialty>Post-Impressionist</specialty>
    <placeOfStudy>Académie Royale des Beaux-Arts</placeOfStudy>
  </artist>
  <!-- Additional artist records follow the same structure -->
</artists>
\end{verbatim}

\subsection{Events}

\begin{itemize}
    \item Robin
\end{itemize}

\begin{verbatim}
<?xml version="1.0" encoding="UTF-8"?>

<!--This is the DTD and the XML document for the events class and is used by two event 
related XQueries-->

<!DOCTYPE events [

<!--The following is a description of the main elements, the cardinality and the general 
outline of the DTD seen below:

Event is set to "*: zero-or-more" as no events are technically needed. There will not 
always be an event planned to document. All of the event elements are set to one as we 
at least need all the information in the document for each single event once. By using
an extra outside bracket for dates, we are not limited in the amount of dates for each 
event, hence why it is "+: one-or-more". There is at least one location needed. Some 
events might take multiple places/rooms but one is the minimum.

The 6 #PCData elements are all elements in the document helping to describe the events.

eventID ATTLIST is #REQUIRED for every new event for easier allocation and logistic 
purposes-->

<!ELEMENT events (event*)>		
<!ELEMENT event (eventName, dates, locations, host, manager, eventDescription)>		
<!ELEMENT dates (date+)>		
<!ELEMENT locations (location+)>	

<!ELEMENT eventName (#PCDATA)>		
<!ELEMENT host (#PCDATA)>
<!ELEMENT date (#PCDATA)>
<!ELEMENT location (#PCDATA)>
<!ELEMENT manager (#PCDATA)>
<!ELEMENT eventDescription (#PCDATA)>

<!ATTLIST event eventID CDATA #REQUIRED>  	

]>

<!--The following XML document is used to give functionality to the events part of our 
UML diagram.

The diagrams showed an external partner working with the admins to create and manage 
events, so in this document the functionality of a system is created,that allows the 
management to oversee certain events with all the necessary details. This could allow 
them to properly prepare, to have a general overview, to structure the year, to 
calculate funds and to make the necessary changes needed.
    
We have the events document, with the individual events all sorted via eventID. Within 
these events, there are multiple other elements that had mostly been included in the 
class diagram already. Some additional events were created. Every event has its eventName, 
its dates, the location it will be occupying/using, the host/external partner, the manager 
responsible for the event and a short event description. All of this information allows 
everyone that has access, to get a quick overview of the most important facts and to 
understand the dimensions of each event.

    There are 8 events with different data to be found below. -->

<events>
    <event eventID="230401001">
        <eventName>Historical Easter Getaway</eventName>
        <dates>
            <date>2023-04-01</date>
        </dates>
        <locations>
            <location>Front Court</location>
        </locations>
        <host>Trinity Food Society</host>
        <manager>Robin Schulz</manager>
        <eventDescription>
The Historical Easter Getaway is an event focused on exploring and celebrating the 
cultural and historical traditions of Easter, featuring themed activities and 
educational presentations.
        </eventDescription>
    </event>
    <!-- Additional events records follow the same structure -->
</events>
    
\end{verbatim}

\subsection{Membership}

\begin{itemize}
    \item Liam
\end{itemize}

\begin{verbatim}
<?xml version="1.0" encoding="UTF-8"?>

<!-- 
    This XML document represents a list of memberships for an organisation or institution. 
    Each membership contains details e.g. purchase price, type, purchase data, expiration, 
    date, visitor ID and membership type
-->

<!DOCTYPE Members [

<!ELEMENT Members (Membership+)> <!-- The '+' cardinality is used to indicate that there 
must be at least one 'Membership' element in the document-->
<!ELEMENT Membership (price, expirationDate, visitorID, type)> <!-- Each 'Membership' 
element contains child elements -->
<!ELEMENT purchasePrice (#PCDATA)>
<!ELEMENT purchaseType (#PCDATA)>
<!ELEMENT purchaseDate (#PCDATA)>
<!ELEMENT expirationDate (#PCDATA)>
<!ELEMENT visitorID (#PCDATA) >
<!ELEMENT type (#PCDATA)>

<!ATTLIST Membership membershipID CDATA #REQUIRED> <!-- The 'membershipID' attribute is 
required for each 'Membership' element to uniquely identify it-->

]>


<Members>
    <!-- Membership record for Visitor ID 1234 with a Patron membership-->
    <Membership membershipID = "0001">
        <purchasePrice>500</purchasePrice>
        <purchaseType>Final</purchaseType>
        <purchaseDate>2023-11-14</purchaseDate>
        <expirationDate>2024-11-14</expirationDate>
        <visitorID>1234</visitorID>
        <type>Patron</type>
    </Membership>
    <!-- Additional membership records follow the same structure -->
</Members>
    
\end{verbatim}

\subsection{Artwork}

\begin{itemize}
    \item Rachel
\end{itemize}

\begin{verbatim}
<?xml version="1.0" ?>

<!-- This XML document represents a museum's collection of artworks. 
        Each artwork entry contains detailed information about the piece,
        including its title, artist, submission date, status, category, and 
        description. -->

<!DOCTYPE museum [

    <!-- The root element 'museum' contains multiple 'artwork' elements. 
            Cardinality '*' is used to allow zero or more artworks. -->
    <!ELEMENT museum (artwork*)>

    <!-- The 'artwork' element represents an individual art piece.
            Each artwork must have exactly one of each of the following child 
            elements: title, artistName, submissionDate, status, category, and 
            description. -->
    <!ELEMENT artwork (title, artistName, submissionDate, status, category, 
    description)>

    <!-- Simple text content (#PCDATA) is expected for each child element. -->
    <!ELEMENT title (#PCDATA)>
    <!ELEMENT artistName (#PCDATA)>
    <!ELEMENT submissionDate (#PCDATA)>
    <!ELEMENT status (#PCDATA)>
    <!ELEMENT category (#PCDATA)>
    <!ELEMENT description (#PCDATA)>

    <!-- The attribute 'artworkID' is optional (IMPLIED) and is used as a 
    unique identifier for each artwork. -->
    <!ATTLIST artwork artworkID CDATA #IMPLIED>
]>

<museum>
    <artwork artworkID="2001">
    <title>Starry Night</title>
    <artistName>Vincent van Gogh</artistName>
    <submissionDate>12/06/2024</submissionDate>
    <status>Displayed</status>
    <category>Post-Impressionism</category>
    <description>
        A depiction of a swirling night sky filled with stars and expressive 
        brushwork that conveys intense emotion.
    </description>
    <!-- Additional artwork entries follow the same structure -->
</museum>
\end{verbatim}

\section{Strengths and Weaknesses}
\subsection{Strengths}

\begin{itemize}
    \item We used a standard to create our document, specifically XML, XPath, and XQuery, which ensured consistency across the project.
    \item We split up the individual XML documents among the team members and then distributed the planned XQueries in a similar way. This distribution had a bias towards those who had worked on certain XML documents that are referenced in specific XQueries.
    \item The use of XQuery functions and modular XML designs allows for easier updates or modifications in the future.
\end{itemize}

\subsection{Weaknesses}

\begin{itemize}
    \item We faced time constraints when designing, which limited our thoroughness.
    \item We gained a lot of knowledge through this project, but we have nowhere to put it, meaning not everyone will immediately understand our work.
    \item We had an enormous amount of classes to pick from in the Class Diagram, making it difficult to choose what to include in our XQueries.
    \item When designing XQueries, we had to reconfigure parts of our Class Diagram so that certain classes were properly connected after integrating them within XQueries.
\end{itemize}
\section{Credits}

\subsection{Finn}
\begin{itemize}
    \item Created 1 XML document: Artist
    \item Added six elements
    \item Created 1 XQuery:
    \begin{description}
        \item [artworkPerArtist:] Outputs the amount of artwork the gallery has per artist.
    \end{description}
    \item Established a GitHub to enable a collaborative workflow to store XML documents.
\end{itemize}

\subsection{Robin}
\begin{itemize}
    \item Created 1 XML document: Events
    \item Added six elements 
    \item Created 2 XQueries:
    \begin{description}
        \item[eventsLengthInDays:] Takes in an event title and prints out the number of days and which days of the month the event occurs.
        \item[eventsByManager:] Takes in a manager's first name and prints out all the events this manager is responsible for.
    \end{description}
    \item Created a poll for task allocation
    \item Created the XML report template
    \item Contributed to:
    \begin{description}
        \item UML Design to XML Implementation
        \item XQueries
        \item XML Documents
    \end{description}
\end{itemize}

\subsection{Liam}
\begin{itemize}
    \item Created 1 XML document: Membership
    \item Added six elements
    \item Created 1 XQuery: 
    \begin{description}
        \item [getMemberRevenue:] Calculates and displays total revenue generated from types of memberships.
    \end{description}
    \item Compiled and formatted the final report
\end{itemize}

\subsection{Rachel}
\begin{itemize}
    \item Created 1 XML document: Artwork
    \item Added two elements
    \item Created 1 XQuery: 
    \begin{description}
        \item[getArtworkForCategory:] Implements user-defined function
    \end{description}
\end{itemize}

\subsection{Nick}
% Add Nick's contributions here if any

\subsection{Roisin}
\begin{itemize}
    \item Created 1 XML document: Donation
    \item Added six elements 
    \item Created 1 XQuery: 
    \begin{description}
        \item[getSpecificDonor:] Retrieves and displays donations by a specific donor.
    \end{description}
    \item Identified strengths and weaknesses of the XML and XQuery design.
\end{itemize}

\subsection{Noah}
\begin{itemize}
    \item Created 1 XML Document: Tickets
    \item Added six elements 
    \item Created 1 XQuery: 
    \begin{description}
        \item[getRevenueFromDate:] Implements user-defined function that retrieves all revenue across Tickets, Membership, and Donations on a certain date.
    \end{description}
    \item Identified strengths and weaknesses of the XML and XQuery design.
\end{itemize}

\newpage
\begin{abstract}
    \item Note from the author: This was my first time writing in Latex - it 
    took me 3 hours to write this document. Please ignore this in the marking, just
    felt like sharing.
\end{abstract}

\end{document}            % Ends the document content